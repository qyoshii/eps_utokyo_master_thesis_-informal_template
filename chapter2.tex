\chapter{研究手法}\label{chap:numerical}
春はあけぼの。
やうやうしろくなりゆく山ぎは、すこしあかりて、紫だちたる雲のほそくたなびきたる。

夏は夜。
月の頃はさらなり、闇もなほ、蛍のおほく飛びちがひたる。
また、ただ一つ二つなど、ほのかにうち光りて行くも、をかし。
雨など降るも、をかし。

秋は夕暮れ。
夕日のさして、山の端いと近くなりたるに、烏の、寝所へ行くとて、三つ四つ、二つ三つなど、飛び急ぐさへ、あはれなり。
まいて、雁などのつらねたるが、いと小さく見ゆるは、いとをかし。
日入りはてて、風の音、虫の音など、はた、言ふべきにあらず。

冬はつとめて。
雪の降りたるは、言ふべきにもあらず。
霜のいと白きも、またさらでも、いと寒きに、火など急ぎおこして、炭持てわたるも、いとつきづきし。
昼になりて、ぬるくゆるびもていけば、火桶の火も、白き灰がちになりて、わろし。
