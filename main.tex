\documentclass[a4j,12pt,twoside,openany]{jreport}

\usepackage{lineheadings}
\input{jdummy.def}
\pagestyle{lineheadings}

\usepackage{lineno}
\usepackage{amsmath, amssymb}
\usepackage{amsfonts}
\usepackage{bm}
\usepackage[dvipdfmx]{graphicx}
\usepackage{graphicx}
\usepackage{verbatim}
\usepackage{makeidx}
\usepackage{dcolumn}
\usepackage{setspace}
\usepackage{comment}
\usepackage{here}
\usepackage{cite}
\usepackage[top=30truemm,bottom=30truemm,left=30truemm,right=30truemm]{geometry}
\usepackage{setspace}
\usepackage{otf}
%\usepackage[square]{natbib}
\renewcommand{\bibname}{引用文献}
\renewcommand{\baselinestretch}{1.}
\renewcommand{\citepunct}{, }
\setlength{\textwidth}{35zw}

%\doublespacing
\begin{document}
% \bibliographystyle{plainnat}
\begin{titlepage}
\begin{spacing}{2.3}

\begin{center}
\vspace*{30truemm}
\fontsize{18pt}{10pt}\selectfont
\textbf{修士論文}\\
\vspace*{30truemm}
\fontsize{20pt}{9pt}\selectfont
\textbf{東京大学地球惑星科学専攻修士論文のタイトル}
\vspace{50truemm}
\fontsize{16pt}{0pt}\selectfont

\textbf{東大 地惑}\\ % 著者
\vspace{15truemm}
\textbf{東京大学大学院理学系研究科}\\ %所属
\vspace{10truemm}
\textbf{地球惑星科学専攻}\\ % 学籍番号
\vspace{10truemm}
\textbf{XXXX科学講座}\\
\vspace{30truemm}

\textbf{令和X年1月XX日~~申請}\\ % 提出日
\end{center}

\end{spacing}
\end{titlepage}
\vspace{10mm}
%%%%%%%%%%%%%%%%%%%%%%%%%%%%%
\newpage
\thispagestyle{empty}

\section*{\fontsize{14pt}{9pt}\selectfont{\textbf{要 旨}}}
月日は百代の過客にして、行き交ふ年もまた旅人なり。船の上に生涯を浮かべ、馬の口とらへて老いを迎ふる者は、日々旅にして旅を栖とす。古人も多く旅に死せるあり。
予もいづれの年よりか、片雲の風に誘はれて、漂泊の思ひやまず、海浜にさすらへ、去年の秋、江上の破屋に蜘蛛の古巣をはらひて、やや年も暮れ、春立てる霞の空に、白河の関越えんと、そぞろ神の物につきて心を狂はせ、道祖神の招きにあひて取るもの手につかず、股引の破れをつづり、笠の緒付けかへて、三里に灸すゆるより、松島の月まづ心にかかりて、住める方は人に譲り、杉風が別所に移るに、草の戸も住み替はる代ぞ雛の家表八句を庵の柱に掛け置く。弥生も末の七日、あけぼのの空朧々として、月は有明にて光をさまれるものから、不二の峰かすかに見えて、上野・谷中の花の梢またいつかはと心細し。むつまじきかぎりは宵よりつどひて舟に乗りて送る。千住といふ所にて舟を上がれば、前途三千里の思ひ胸にふさがりて、幻のちまたに離別の涙をそそぐ。行く春や鳥なき魚の目は涙これを矢立の初めとして行く道なほ進まず。人々は途中に立ち並びて、後ろ影の見ゆるまではと見送るなるべし。

%%%%%%%%%%%%%%%%%%%%%%%%%%%%%
\newpage
\thispagestyle{empty}

\section*{\fontsize{14pt}{9pt}\selectfont{\textbf{Abstract}}}
Animals are happy so long as they have health and
enough to eat. Human beings, one feels...

%%%%%%%%%%%%%%%%%%%%%%%%%%%%%
\newpage
\tableofcontents
\clearpage

\newpage
%%%%%%%%%%%%%%%%%%%%%%%%%%%%%
\part{序論}
\chapter{Introduction}\label{chap:intro}
%%%%%%%%%%%%%%%%%%%%%%%%%%%%%%%%%%%%%%%%%%%%%%%%%%%%%%%%%%%%
\section{研究背景}
貸そうかな。まあ、あてにすんなひどすぎる借金 
\subsection{観測・室内実験}

\subsection{数値計算}
引用例\cite{yoshii2019delamination, yajima2020aversion}
\subsection{理論モデル}

対数正規分布は以下の関数で
\begin{align*}
   f(x) = \dfrac{1}{\sqrt{2\pi}\sigma x} \exp\left( -\dfrac{(\ln x -\mu)^2}{2\sigma^2} \right)
\end{align*}
とかける。

\input{chapter2}
\part{本論}
\chapter{結果}\label{chap:result}

\begin{figure}[h]
\centering
\includegraphics[width=0.75\textwidth]{eri.png}
\caption{\label{fig:z-1}ERI, UTokyo}
\end{figure}
\chapter{考察}\label{chap:discussion}
じゅげむ じゅげむ ごこうのすりきれ
かいじゃりすいぎょの すいぎょうまつ
うんらいまつ ふうらいまつ
くうねるところに すむところ
やぶらこうじの ぶらこうじ
パイポパイポ
パイポのシューリンガン
シューリンガンのグーリンダイ
グーリンダイのポンポコピーのポンポコナーの
ちょうきゅうめいのちょうすけ
%%%%%%%%%%%%%%%%%%%%%%%%%%%%%
\part{結論}
\chapter{結論と今後の展望}\label{chap:conclusion}

%%%%%%%%%%%%%%%%%%%%%%%%%%%%%
\chapter*{謝辞}
\addcontentsline{toc}{chapter}{謝辞}
子曰、「吾十有五にして学に志す。三十にして立つ。四十にして惑はず。五十にして天命を知る。六十にして耳順ふ。七十にして心の欲する所に従へども、矩を踰えず」。

%%%%%%%%%%%%%%%%%%%%%%%%%%%%%
\appendix
%%%%%%%%%%%%%%%%%%%%%%%%%%%%%
\chapter{計算の補足とか}\label{chap:a}
%%%%%%%%%%%%%%%%%%%%%%%%%%%%%
\chapter{補遺の場所}\label{chap:b}
%%%%%%%%%%%%%%%%%%%%%%%%%%%%%
\chapter{おまけ}\label{chap:c}

%%%%%%%%%%%%%%%%%%%%%%%%%%%%%
\addcontentsline{toc}{chapter}{\bibname}
\bibliographystyle{jgr}
\bibliography{ref.bib}
\end{document}